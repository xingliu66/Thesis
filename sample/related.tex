\chapter{Related Work} \label{chap:related}


\section{Live streaming and 360\degree{} video streaming}
First, to the best of our knowledge, few previous work, if any, have
measured or analyzed 360\degree{} live video streaming. However, there
is a large corpus of work in two related areas: \emph{personalized live streaming} and \emph{360\degree{} video streaming}.

\textbf{Living Streaming} broadcasts users themselves
and their surroundings using their (mobile) devices; viewers all
over the world can watch the live feed in real time. In 2016, two
papers~\cite{siekkinen2016first,wang2016anatomy} simultaneously studied Periscope and Meerkat, back
then the most popular platforms for personalized streaming on Android
and iOS mobile devices. Although different in their methodologies,
these two papers have a similar goal: shed some light on the
architecture (\emph{e.g.}, protocols and settings), the scale (\emph{e.g.}, the number
of broadcasters), and the performance (\emph{e.g.}, video quality and stalls)
of both streaming platforms. Our paper shares a similar goal but
in the context of 360\degree{} live video streaming offered by two different
platforms, YouTube and Facebook. Accordingly, our methodology in \lime
largely departs from the approaches proposed in~\cite{siekkinen2016first,wang2016anatomy}, as well
as our observations. Another recent paper~\cite{tang2016meerkat} studied the content and human
factors for Periscope and Meerkat.

Twitch is another popular platform for personalized live streaming,
with its primary focus on gaming broadcasting. Twitch differs
from Periscope and Meerkat since its broadcasters are mostly not
mobile. Pires \emph{et al}. \cite{pires2015youtube} were the first to look into Twitch (and
YouTube Live) back in 2015. Using a three-month dataset, they
showed the importance of Twitch with traffic peaks at more than
1 Tbps and millions of uploaded videos. Also in 2015, Zhang \emph{et al}. \cite{zhang2015crowdsourced} preliminarily investigated Twitch’s infrastructure using
both crawled data and captured traffic of local broadcasters/viewers.
More recently, Deng \emph{et al}. \cite{deng2017internet} expanded the latter study by exploring
Twitch’s infrastructure via a network of free proxies located
worldwide. They identified a geo-distributed infrastructure with
fine-grained server allocations, i.e., resources are dynamically allocated
to live streaming events based on their (growing) popularity.
There are also some earlier measurements on other live
streaming platforms such as live Internet TV and P2P live streaming
\cite{hei2007measurement,kaytoue2012watch,li2011measurement,silverston2007measuring,sripanidkulchai2004analysis}. The scope of \lime largely departs from them since we
focus on different content (360\degree{} live videos) and platforms (YouTube
and Facebook).

\textbf{360\degree{} Video Streaming} has become a hot research topic recently.
Researchers have investigated multiple aspects including projection/encoding methods \cite{kuzyakov2016next,kuzyakov2015under,lee2016rich360,nasrabadi2017adaptive,zhou2017measurement}, energy consumption \cite{jiang2017power},
viewport-adaptive streaming \cite{almquist2018prefetch,bao2016shooting,corbillon2017optimal,corbillon2017viewport,graf2017towards,petrangeli2017http,qian2016optimizing,qian2018flare,xiao2017optile,xie2017360probdash,xie2018cls},
cross-layer interaction \cite{sun2018multi,xie2017poi360}, and user experience \cite{broeck2017s}, \emph{etc}. Most
of the above studies focused on non-live 360\degree{} videos and none of
them investigated commercial 360\degree{} video streaming platforms as
we have done using crowd-sourcing.

In 2017, Afzal \emph{et al}. \cite{afzal2017characterization} studied the characteristics of (non-live)
360\degree{} videos uploaded to YouTube by simply searching for such
videos using keywords like ``360''. By analyzing a dataset of 4570
videos, they found that compared to regular videos, 360\degree{} videos
tend to be shorter, having higher resolutions, and more static (less
motion). \lime complements this effort since we focus on the
streaming performance rather than the 360\degree{} video characteristics. Furthermore, \lime investigates live streaming and expand our analysis
to Facebook as well.

In 2017, our positioning workshop paper
conducted a preliminary investigation of 360\degree{} live videos \cite{liu2017360}. \lime goes beyond \cite{liu2017360} by making several contributions: developing
a holistic measurement system for live videos, conducting crowdsourced
measurements for live 360\degree{} videos, and quantifying the
benefits of several key optimizations.

\section{Mobile VR}
In Chapter~\S\ref{chap:vr} we design and implement a multi-user mobile VR system \firefly. In literature relevant systems have been proposed, however they are all tested under certain single user scenarios, and there lack systematic study on its
multi-user counterpart.

\textbf{Single-user Mobile VR} has been well investigated in literature.
Furion \cite{lai2019furion} and Flashback \cite{boos2016flashback} demonstrate
high-quality single-user VR on COTS smartphones.
MoVR \cite{abari2016cutting,abari2017enabling} employs 60 GHz mmWave wireless for mobile
VR. Liu \emph{et al.} \cite{liu2018cutting} proposed system-level optimizations
for the mobile VR rendering pipeline. Tan \emph{et al.} explored supporting
mobile VR over LTE \cite{tan2018vr}. None of the above work
explicitly focuses on the multi-user scenario.

\textbf{Multi-user VR/AR}. Despite a plethora of work on single user
VR, much fewer studies have been conducted on
multi-user environment. The most relevant work to \firefly is
MUVR \cite{li2018muvr} which can only be simulated to support 4 concurrent VR users.
Bao \emph{et al.} developed
a multi-user 360° video streaming system based on
multicast \cite{bao2017motion}. A recent positioning paper \cite{liu2019supporting} discusses several
practical issues of designing a multi-user VR system
(without system implementation). Some studies investigated
multi-user or collaborative augmented reality (AR) \cite{qiu2018avr,ran2019sharear,zhang2018cars}.
Compared to the above work, \firefly is a generic multi-user
VR system. It achieves much better scalability compared to
MUVR.


\section{Characterizing Android Wear OS}

\textbf{Mobile devices ``in the wild''}. Researchers have carried out numerous
crowd-sourced measurements of smartphones~\cite{huang10,falaki10_mobisys,shepard10,phonelab,qian12_mobisys,rosen15_imc,nikravesh15_mobisys,chen15:sigmetrics}. While they partially motivate our smartwatch user study, much fewer efforts have been made for wearable devices.
Among them,
Gouveia \etal studied how users engage with activity trackers~\cite{gouveia15_ubicomp}.
%
Lazar \etal interviewed 17 participants to study the incentives of using and abandoning smart devices~\cite{lazar15_ubicomp}.
%
Lyons studied usage practices of traditional dumb watches by conducting a user survey~\cite{lyons15_iswc}.
%
Min \etal characterized smartwatch battery use based on online survey and a user study involving 17 users~\cite{min15_iswc}. They studied users' satisfaction toward smartwatch life, charging behaviors,
and interaction patterns.
%
In a recent extended abstract~\cite{poyraz_iiswc16}, Poyraz \etal also described their smartwatch user study involving 32 users for 70 days. Using the collected data, they analyzed the watches' power consumption and characterized user activities. While the detailed data collection and measurement methodologies were not fully documented, many of their findings such as the active state's high energy contribution compared to its short usage duration are qualitatively similar to ours.
%
Compared to~\cite{min15_iswc} and~\cite{poyraz_iiswc16}, we instead investigate a much wider spectrum of characteristics such as push notification, app usage, and network traffic by leveraging a much richer set of data items collected in Table~\ref{tab:data}.

\textbf{Android Wear OS}. Recently, Liu \etal analyzed the execution of Android Wear OS and presented a series of inefficiencies and OS implications~\cite{liu16_mobisys,liu15_apsys}.
Compared to its in-lab, controlled experiments, our study on Android smartwatch contributes the understanding of real-world wearable usage as well as the entailed design implications -- backed by more solid evidence.
Furthermore, our study characterizes key system aspects that were missing in the prior work, most notably power modeling and network behaviors.

\textbf{Other wearable system and applications} have also been studied in literature.
LiKamWa \etal characterized the energy consumption of Google Glass~\cite{likamwa13_apsys}.
%
Huang \etal proposed a fast storage system for wearables based on battery-backed RAM and offloading~\cite{huang15_atc}.
%
Santagati \etal designed an ultrasonic networking framework for wearable
medical devices~\cite{santagati15_mobisys}.
%
Miao \etal investigated the implications of smartwatches' circular display on resource usage~\cite{miao16_hotmobile}.
%
Ham \etal proposed a novel display energy conservation scheme for wearables~\cite{ham15_atc}.
%
There also exists a large body of work from the mobile computing, sensing, and HCI community. These studies focus on novel wearable applications~\cite{shen16_mobisys, nirjon15_mobisys,mayberry14_mobisys}, wearable user interface design~\cite{chen14_chi, plaumann16_iswc}, and wearable security~\cite{wang15_mobicom, liu15_ccs}.
%
In contrast, our smartwatch measurement focuses on characterizing wearable user behavior, application usage, energy consumption, and network traffic -- all in the wild.

